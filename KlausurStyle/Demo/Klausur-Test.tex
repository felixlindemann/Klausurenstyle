\NeedsTeXFormat{LaTeX2e}
\documentclass{scrartcl}
 
\usepackage[
		%UseSVN, 			%Soll SVN als Versionsverwaltung genutzt werden?
		%answers, 			%L�sungF�rStudenten
		MusterLoesung,		%Musterl�sung
		isGruppenarbeit,	%Gruppenarbeit
		GiveLines,			%Raum f�r Antworten
		GivePoints,			%K�sten zur Benotung
		Inhaltsverzeichnis 	%Inhaltsverzeichnis
	]
	{../Klausur}
% Setup-Paramter	  

\pgfdeclareimage[height=1.5cm]{logo}{grafik/logo.png} 
\setUniversity{Name of University}
\setKlausurVersionDatum{\today~\thistime~Uhr}
\setKlausurTyp{Tutorium}
\setKlausurName{\KlausurTyp~No.5 - Travelling Salesman Problem}
\setKlausurFach{Transportwirtschaft Sommersemester 2011}
\setKlausurDatum{16. Mai 2011 09:30 Uhr}
\setKlausurProfessorA{\small\textbf{Prof. Dr. Max Mustermann} \\\footnotesize max.mustermann@hs-neu-ulm.de}
\setKlausurProfessorB{\small\textbf{Dipl. Kfm. Felix Lindemann} \\\footnotesize felix.lindemann@hs-neu-ulm.de}
\setKlausurProfessorC{\small\textbf{Dipl. Kfm. Dagmar Hase} \\\footnotesize dagmar.hase@hs-neu-ulm.de}
%
\bibliography{bib/literature}

\addtocategory{primary}{  Domschke.1997,Domschke.2005,Durr.1992.750,Kathofer.2008.740}    
\addtocategory{secondary}{   Feige.2008,LutzWestphal.2007.751,Neumann.2004}    
 
\begin{document} 
	\sffamily\onehalfspacing
	%Falls kein Subversion verwendet wird, kann diese Zeile ggf. manuelle angepasst werden. 
	\TitleTutoriumA  
\Aufgabe %1
{Theoretische Fragen - Problemmodell I}
{
	Tipp: Zur Bearbeitung der folgenden Aufgaben ist ein 
	Blick in die empfohlene Literatur sowie die 
	Vorlesungsunterlagen notwendig!
} 
%%%%%%%%%%%%%%%%%%%%%%%%%%%%%%%%%%%%%%%%%%%%%%%%%%%%%%%%%%%%%%%%%%%%%%%%%%%%%%%%%%%%%%%%%%%%%%%%
%%%%%%%%%%%%%%%%%%%%%%%%%%%%%%%%%%%%%%%%%%%%%%%%%%%%%%%%%%%%%%%%%%%%%%%%%%%%%%%%%%%%%%%%%%%%%%%%
%%%%%%%%%%%%%%%%%%%%%%%%%%%%%%%%%%%%%%%%%%%%%%%%%%%%%%%%%%%%%%%%%%%%%%%%%%%%%%%%%%%%%%%%%%%%%%%%
\TeilAufgabe
{Beschreiben Sie das Travelling Salesman Problem in eigenen Worten.}
{4}
\KlausurAntwortKasten{4cm}
\KlausurErgebnis{Siehe Vorlesungsunterlagen - Tutorium}
\KlausurMusterLoesung{
	Finde die kostenminimale Rundreise(1), 
	die am gleichen Punkt startet(1) und endet(1) und 
	jeden Ort nur einmal besucht.(1)
} 
\KlausurKorrekturhinweis{4p.} 
\KlausurErlaeuterung{-}
%%%%%%%%%%%%%%%%%%%%%%%%%%%%%%%%%%%%%%%%%%%%%%%%%%%%%%%%%%%%%%%%%%%%%%%%%%%%%%%%%%%%%%%%%%%%%%%%
%%%%%%%%%%%%%%%%%%%%%%%%%%%%%%%%%%%%%%%%%%%%%%%%%%%%%%%%%%%%%%%%%%%%%%%%%%%%%%%%%%%%%%%%%%%%%%%%
%%%%%%%%%%%%%%%%%%%%%%%%%%%%%%%%%%%%%%%%%%%%%%%%%%%%%%%%%%%%%%%%%%%%%%%%%%%%%%%%%%%%%%%%%%%%%%%%
\TeilAufgabe
{Geben Sie das formale Problem Modell f�r das Traveling Salesman Problem (TSP) an.}
{10}
\KlausurAntwortKasten{8cm}
\KlausurErgebnis{Siehe Vorlesungsunterlagen - Tutorium}
\KlausurMusterLoesung{
	\begin{align}
	\min f(x) &= \sum\limits_{i=1}^n\sum\limits_{j=1}^n c_{ij} \cdot x_{ij} 
		& \label{eq.TSP.Zielfunktion}\\
	s.t. & \nonumber\\
	\sum\limits_i x_{ij} &= 1 & \forall ~j= 1 \ldots n
		\label{eq.TSP.cond1}\\
	\sum\limits_j  x_{ij} &= 1 & \forall ~i= 1 \ldots n
		\label{eq.TSP.cond2}\\
	x_{ij} &\in \left\{0,1\right\} & \forall ~i,j= 1 \ldots n
		\label{eq.TSP.cond3}\\
	\sum\limits_{i\in Q}\sum\limits_{j\in Q} x_{ij} &\leq |Q| -1 & 
		\forall Q\text{~}\subset \text{~}V \text{ with } 2\leq |Q| 
		\leq \left\lfloor \frac{n}{2}\right\rfloor
			\label{eq.TSP.cond4}
\end{align}
} 
\KlausurKorrekturhinweis{je korrekter Zeile 2p max. 10p} 
\KlausurErlaeuterung{-}
%%%%%%%%%%%%%%%%%%%%%%%%%%%%%%%%%%%%%%%%%%%%%%%%%%%%%%%%%%%%%%%%%%%%%%%%%%%%%%%%%%%%%%%%%%%%%%%%
%%%%%%%%%%%%%%%%%%%%%%%%%%%%%%%%%%%%%%%%%%%%%%%%%%%%%%%%%%%%%%%%%%%%%%%%%%%%%%%%%%%%%%%%%%%%%%%%
%%%%%%%%%%%%%%%%%%%%%%%%%%%%%%%%%%%%%%%%%%%%%%%%%%%%%%%%%%%%%%%%%%%%%%%%%%%%%%%%%%%%%%%%%%%%%%%%
	\newpage
	 
\Aufgabe %2
	{Anwendung des TSP-Problem Modells}
	{
		\minisec{Bearbeitungshinweise zur Anwendung des TSP Problem-Modells}
			Pr�fen Sie f�r die gegebenen Graphen anhand der \textbf{formalen} 
			Bedingungen des TSP-Problem-Modells, ob die eingezeichnete 
			L�sung g�ltig ist.\\
			Es gen�gt eine nicht erf�llte Bedingung, um zu zeigen, 
			dass die L�sung ung�ltig ist; um zu zeigen, dass eine L�sung 
			g�ltig ist, m�ssen \emph{alle} Bedingungen erf�llt sein.
	} 
%				 
\TeilAufgabe{
	\label{aufg.2.a}
	Ist die L�sung aus Abbildung \ref{fig.Graph.TSP.2.a} g�ltig?
	\begin{figure}[H]
		\centering
		% \svnInfo $Id: Klausur.sty 2 2011-05-18 08:35:27Z felixlindemann $
		\begin{tikzpicture}[ shorten >=1pt, node distance=2.5cm,semithick,scale=0.75 ]
			\tikzstyle{every state}=[fill=white,draw,circle,inner sep=1pt]
			\node[state](A)                    			{$1$};
			\node[state]         (B) [below left of=A] 			{$2$};
			\node[state]         (C) [left of=B] 			{$3$};
			\node[state]         (D) [above left of=C] 						{$4$};
			\node[state]         (E) [above of=D] 			{$5$};	
			\node[state]         (F) [left of=E] 			{$6$};	
			\node[state]         (G) [right of=E] 			{$7$};	
			\tikzstyle{tikzNodeLength}=[fill=white,draw,circle,inner sep=1pt,font=\sf \tiny]								
			\begin{scope}[>=latex,thick]
				\path[->] (A) edge	(B);
				\path[->] (B) edge	(C);											 
				\path[->] (E) edge	(D);
				\path[->] (D) edge  	(C);
			%		\path[->] (C) edge  	(E);
				\path[->] (E) edge	(G);
				\path[->] (G) edge	(A);
			\end{scope}
		\end{tikzpicture} 
		\caption{Graph zu Aufgabe \ref{aufg.2.a}}
		\label{fig.Graph.TSP.2.a}
	\end{figure}
}
{2} % 2 Punkte
\KlausurAntwortKasten{10cm}
\KlausurErgebnis{Siehe Vorlesungsunterlagen - Tutorium}
\KlausurMusterLoesung{$\sum\limits_{i}x_{ij} =0 $ f�r $j =6$ (1p) $\rightarrow$ ung�ltig (1p)} 
\KlausurKorrekturhinweis{2p.} 
\KlausurErlaeuterung{-}
%%%%%%%%%%%%%%%%%%%%%%%%%%%%%%%%%%%%%%%%%%%%%%%%%%%%%%%%%%%%%%%%%%%%%%%%%%%%%%%%%%%%%%%%%%%%%%%%
%%%%%%%%%%%%%%%%%%%%%%%%%%%%%%%%%%%%%%%%%%%%%%%%%%%%%%%%%%%%%%%%%%%%%%%%%%%%%%%%%%%%%%%%%%%%%%%%
%%%%%%%%%%%%%%%%%%%%%%%%%%%%%%%%%%%%%%%%%%%%%%%%%%%%%%%%%%%%%%%%%%%%%%%%%%%%%%%%%%%%%%%%%%%%%%%%

	\KlausurFinalize 
\end{document}
