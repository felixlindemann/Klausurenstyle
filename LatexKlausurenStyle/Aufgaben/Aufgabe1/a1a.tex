% \svnInfo $Id: a1a.tex 2 2011-05-18 10:04:02Z felixlindemann $

\TeilAufgabe
				 {Auf \url{http://de.wikipedia.org/wiki/Dijkstra-Algorithmus} hei�t es:\\
"`\emph{Der Algorithmus von Dijkstra [\ldots] ist ein Algorithmus aus der Klasse der Greedy-Algorithmen 
und dient der Berechnung eines k�rzesten Pfades zwischen einem Startknoten 
und einem beliebigen Knoten in einem kantengewichteten Graphen.}"'\\
Diskutieren Sie, ob der Algorithmus von Dijkstra tats�chlich zur Familie der Greedy-Heuristiken geh�rt.\\
Hinweis: Definieren Sie dazu zun�chst den Begriff der Greedy Heuristik.}
				 {6}
\KlausurAntwortLinien{6}
\KlausurErgebnis{Richtig, Erl�uterungen siehe Tutorium}
\KlausurMusterLoesung{Richtig/ja(1p), eine Greedyheuristik w�hlt \emph{gierig} die lokal beste L�sung aus(1p), ohne dabei auf das Gesamtoptimum zu achten.(1p)
Der Alg. von Dijkstra w�hlt auch das lokale Optimum aus.(1p) Da ein gesamt-k�rzester Weg immer aus Teil-k�rzesten Wegen besteht,(1p)
kann der Alg. von Dijkstra als Sonderfall interpretiert werden.(1p)} 
\KlausurKorrekturhinweis{je Aspekt 1p, Gro�z�gig bewerten } 
\KlausurErlaeuterung{-}
%
\TeilAufgabe
				 {Gehen Sie davon aus, dass in einem beliebigen symmetrischen Netzwerk 5 Knoten ($A-E$) existieren.\\
Diskutieren Sie, ob der k�rzeste Weg von $A$ nach $E$ im Netz identisch mit dem Weg von $E$ nach $A$ ist, 
wenn beide Wege mit dem \emph{Algorithmus von Dijkstra} berechnet werden.\\
Illustrieren Sie Ihre Ausf�hrungen mit einer Skizze.}
				 {4}
\KlausurAntwortLinien{4}
\KlausurAntwortKasten{4cm}
\KlausurErgebnis{Ja, beide Wege sind gleich-lang, Erl�uterungen siehe Tutorium}
\KlausurMusterLoesung{Ja, beide Wege sind gleich-lang.(1p) Im Falle eines asymmetrischen (1P) Netzwerks kann sich eine andere Route (1p) ergeben.} 
\KlausurKorrekturhinweis{F�r korrekte Antwort 3P, 1p Skizze} 
\KlausurErlaeuterung{-}
%%
%\ 
\ifisAufgabenstellung{\newpage}
\TeilAufgabe
				 {Was ist die Bedeutung von $Q$ im Dijkstra-Tableau? }
				 {2}
\KlausurAntwortLinien{6}
\KlausurErgebnis{Richtig, Erl�uterungen siehe Tutorium}
\KlausurMusterLoesung{Die Menge $Q$ ist eine sortierte Liste der erreichbaren Knoten (1p).
 Die Knoten sind aufsteigend sortiert anzugeben (1p). \\Alternativ: Das Element mit niedrigstem Kostenwert steht vorne.(1p)   }
\KlausurKorrekturhinweis{je Aspekt 1p} 
\KlausurErlaeuterung{-}
% 
\TeilAufgabe
				 {Wieviele Knoten werden beim Dijkstra-Algorithmus bei der Ermittlung von k�rzesten Wegen
				 je Iteration ausgew�hlt?}
				 {1}
\KlausurAntwortLinien{2}
\KlausurErgebnis{genau 1 Knoten}
\KlausurMusterLoesung{genau 1 Knoten(1p)  } 
\KlausurKorrekturhinweis{1p} 
\KlausurErlaeuterung{-}
% 