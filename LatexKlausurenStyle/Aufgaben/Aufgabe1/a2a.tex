% \svnInfo $Id: a2a.tex 2 2011-05-18 10:04:02Z felixlindemann $
\Aufgabe %1
				 {Korrektur Fragen}
				 {\ifisAufgabenstellung{Im Folgenden erhalten Sie den K�rzeste-Wege-L�sungsvorschlag 
				 (Tabelle \ref{tbl.SolutionStudent}) eines Kommilitonen f�r nachstehenden Graphen.
				 \input{grafik/netzdijkstra}
				 % \svnInfo $Id: a2TableStud.tex 2 2011-05-18 10:04:02Z felixlindemann $
\begin{table}[H]\centering
\begin{tabular}{|c||cc|cc|cc|cc|cc|cc|cc|}
\hline
 & \multicolumn{2}{|c|}{1}   & \multicolumn{2}{|c|}{2} &  \multicolumn{2}{|c|}{3} 
 & \multicolumn{2}{|c|}{4} &  \multicolumn{2}{|c|}{5}    & \multicolumn{2}{|c|}{6} &  \multicolumn{2}{|c|}{7}\\  
 & d & p & d & p & d & p & d & p & d & p & d & p & d & p\\\hline\hline
1 & 0 & - & 0 & - & 0 & - & 0 & - & 0 & - & 0 & - & 0 & -\\\hline
2 & $\infty$ & - & 4 & 1 & 4 & 1 & 4 & 1 & 4 & 1 & 4 & 1 & 4 & 1\\\hline
3 & $\infty$ & - & $\infty$ & 1 & 7 & 2 & 7 & 2 & 7 & 2 & 7 & 2 & 7 & 2\\\hline
4 & $\infty$ & - & $\infty$ & - & 8 & 2 & 8 & 2 & 8 & 2 & 8 & 2 & 8 & 2\\\hline
5 & $\infty$ & - & $\infty$ & - & 6 & 2 & 6 & 2 & 6 & 2 & 6 & 2 & 6 & 2\\\hline
6 & $\infty$ & - & $\infty$ & - & $\infty$ & - & 13 & 5 & 12 & 4 & 12 & 4 & 12 & 4\\\hline\hline
$Q$ & \multicolumn{2}{|c|}{ }    & \multicolumn{2}{|c|}{ }    & \multicolumn{2}{|c|}{ }    & \multicolumn{2}{|c|}{ }   & \multicolumn{2}{|c|}{ }   & \multicolumn{2}{|c|}{ }    & \multicolumn{2}{|c|}{ }   
\\\hline
\end{tabular}
\caption{L�sung Ihres Kommilitonen}
\label{tbl.SolutionStudent}
\end{table}}	~			 
				 }
\TeilAufgabe{Stellen Sie fest, ob Ihr Kommilitone richtig gerechnet hat. Berechnen Sie dazu 
				 eine eigene L�sung.
				 \ifisAufgabenstellung{ Nutzen Sie f�r Ihre L�sung Tabelle \ref{tbl.SolutionYou}!}}
				 {15}
				 \ifisAufgabenstellung{
% \svnInfo $Id: a2TableYour.tex 2 2011-05-18 10:04:02Z felixlindemann $
\begin{table}[H]\centering
\begin{tabular}{|c||p{0.75cm}|p{0.75cm}|p{0.75cm}|p{0.75cm}|p{0.75cm}|p{0.75cm}|p{0.75cm}|p{0.75cm}|p{0.75cm}|p{0.75cm}|p{0.75cm}|p{0.75cm}|p{0.75cm}|p{0.75cm}|}
\hline
 & \multicolumn{2}{|c|}{1}   & \multicolumn{2}{|c|}{2} &  \multicolumn{2}{|c|}{3} 
 & \multicolumn{2}{|c|}{4} &  \multicolumn{2}{|c|}{5}    & \multicolumn{2}{|c|}{6} &  \multicolumn{2}{|c|}{7}\\  
 & d & p & d & p & d & p & d & p & d & p & d & p & d & p\\\hline\hline
1 &   &   &   &   &   &   &   &   &   &   &   &   &   &  \\[1em]\hline
2 &   &   &   &   &   &   &   &   &   &   &   &   &   &  \\[1em]\hline
3 &   &   &   &   &   &   &   &   &   &   &   &   &   &  \\[1em]\hline
4 &   &   &   &   &   &   &   &   &   &   &   &   &   &  \\[1em]\hline
5 &   &   &   &   &   &   &   &   &   &   &   &   &   &  \\[1em]\hline
6 &   &   &   &   &   &   &   &   &   &   &   &   &   &  \\[1em]\hline
$Q$ & \multicolumn{2}{|c|}{ }    & \multicolumn{2}{|c|}{ }    & \multicolumn{2}{|c|}{ }    & \multicolumn{2}{|c|}{ }   & \multicolumn{2}{|c|}{ }   & \multicolumn{2}{|c|}{ }    & \multicolumn{2}{|c|}{ }   
\\[1em]\hline
\end{tabular}
\caption{Tragen Sie hier Ihre L�sung ein.}
\label{tbl.SolutionYou}
\end{table}	}	
\KlausurErgebnis{% \svnInfo $Id: a2TableML.tex 2 2011-05-18 10:04:02Z felixlindemann $
\begin{table}[H]\centering
\begin{tabular}{|c||cc|cc|cc|cc|cc|cc|cc|}
\hline
 & \multicolumn{2}{|c|}{1}   & \multicolumn{2}{|c|}{2} &  \multicolumn{2}{|c|}{3} 
 & \multicolumn{2}{|c|}{4} &  \multicolumn{2}{|c|}{5}    & \multicolumn{2}{|c|}{6} &  \multicolumn{2}{|c|}{7}\\  
 & d & p & d & p & d & p & d & p & d & p & d & p & d & p\\\hline\hline
1 & 0 & - & 0 & - & 0 & - & 0 & - & 0 & - & 0 & - & 0 & -\\\hline
2 & $\infty$ & - & 4 & 1 & 4 & 1 & 4 & 1 & 4 & 1 & 4 & 1 & 4 & 1\\\hline
3 & $\infty$ & - & 8 & 1 & 7 & 2 & 7 & 2 & 7 & 2 & 7 & 2 & 7 & 2\\\hline
4 & $\infty$ & - & $\infty$ & - & 8 & 2 & 8 & 2 & 8 & 2 & 8 & 2 & 8 & 2\\\hline
5 & $\infty$ & - & $\infty$ & - & $\infty$ & - & 14 & 3 & 11 & 4 & 11 & 4 & 11 & 4\\\hline
6 & $\infty$ & - & $\infty$ & - & $\infty$ & - & $\infty$ & - & 12 & 4 & 12 & 4 & 12 & 4\\\hline\hline
$Q$ & \multicolumn{2}{|c|}{$\left\{1\right\}$}    & \multicolumn{2}{|c|}{$\left\{2,3\right\}$}    & \multicolumn{2}{|c|}{$\left\{3,4\right\}$}    & \multicolumn{2}{|c|}{$\left\{4,5\right\}$}   & \multicolumn{2}{|c|}{$\left\{5,6\right\}$}   & \multicolumn{2}{|c|}{$\left\{6\right\}$}    & \multicolumn{2}{|c|}{$\left\{-\right\}$}   
\\\hline
\end{tabular}
\caption{L�sung}
\label{tbl.Solution}
\end{table}		}
\KlausurMusterLoesung{% \svnInfo $Id: a2TableML.tex 2 2011-05-18 10:04:02Z felixlindemann $
\begin{table}[H]\centering
\begin{tabular}{|c||cc|cc|cc|cc|cc|cc|cc|}
\hline
 & \multicolumn{2}{|c|}{1}   & \multicolumn{2}{|c|}{2} &  \multicolumn{2}{|c|}{3} 
 & \multicolumn{2}{|c|}{4} &  \multicolumn{2}{|c|}{5}    & \multicolumn{2}{|c|}{6} &  \multicolumn{2}{|c|}{7}\\  
 & d & p & d & p & d & p & d & p & d & p & d & p & d & p\\\hline\hline
1 & 0 & - & 0 & - & 0 & - & 0 & - & 0 & - & 0 & - & 0 & -\\\hline
2 & $\infty$ & - & 4 & 1 & 4 & 1 & 4 & 1 & 4 & 1 & 4 & 1 & 4 & 1\\\hline
3 & $\infty$ & - & 8 & 1 & 7 & 2 & 7 & 2 & 7 & 2 & 7 & 2 & 7 & 2\\\hline
4 & $\infty$ & - & $\infty$ & - & 8 & 2 & 8 & 2 & 8 & 2 & 8 & 2 & 8 & 2\\\hline
5 & $\infty$ & - & $\infty$ & - & $\infty$ & - & 14 & 3 & 11 & 4 & 11 & 4 & 11 & 4\\\hline
6 & $\infty$ & - & $\infty$ & - & $\infty$ & - & $\infty$ & - & 12 & 4 & 12 & 4 & 12 & 4\\\hline\hline
$Q$ & \multicolumn{2}{|c|}{$\left\{1\right\}$}    & \multicolumn{2}{|c|}{$\left\{2,3\right\}$}    & \multicolumn{2}{|c|}{$\left\{3,4\right\}$}    & \multicolumn{2}{|c|}{$\left\{4,5\right\}$}   & \multicolumn{2}{|c|}{$\left\{5,6\right\}$}   & \multicolumn{2}{|c|}{$\left\{6\right\}$}    & \multicolumn{2}{|c|}{$\left\{-\right\}$}   
\\\hline
\end{tabular}
\caption{L�sung}
\label{tbl.Solution}
\end{table} } 
\KlausurKorrekturhinweis{je Korrekter Iteration jeweils 
1p f�r korrektes $Q$, 
1p f�r korrekte Entfernungen
Max. 14p.} 
\KlausurErlaeuterung{-}
\TeilAufgabe{Erkl�ren Sie Ihrem Kommilitonen die gemachten Fehler.}
				 {3}
\KlausurAntwortLinien{4}
\KlausurErgebnis{\begin{enumerate}
	\item $Q$ wurde pauschal nicht angegeben
	\item $\infty$ in Iteration 2 muss eine 8 sein
	\item Die Strecke von $v_2$ nach $v_5$ die in Iteration 3 verwendet wurde 
				existiert nur in der Gegenrichtung.
\end{enumerate}}
\KlausurMusterLoesung{\begin{enumerate}
	\item $Q$ wurde pauschal nicht angegeben
	\item Die Strecke von $v_2$ nach $v_5$ die in Iteration 3 verwendet wurde 
				existiert nur in der Gegenrichtung.
\end{enumerate} } 
\KlausurKorrekturhinweis{jeweils 1p Max 2} 
\KlausurErlaeuterung{-}
\TeilAufgabe{Erkl�ren Sie Ihrem Kommilitonen wie aus Ihrem Dijkstra Tableau 
					 \ifisAufgabenstellung{(Tabelle \ref{tbl.SolutionYou})}
						der k�rzeste Weg sowie die L�nge von Knoten $v_1$ zu Knoten $v_5$ abgelesen werden kann. 
						Geben Sie dabei auch den Weg, sowie die Gesamtl�nge an!}
				 {5}
\KlausurAntwortLinien{4}
\KlausurErgebnis{Rekursiv - Siehe Tutorium}
\KlausurMusterLoesung{Rekursiv(1p): Der Weg vom Startknoten zu Knoten $v_5$ betr�gt 11 Einheiten(1p) (Zeile 5 Iteration 7 Spalte d). Der Vorg�nger von Knoten $v_5$ ist $v_4$(Zeile 5 Iteration 7 Spalte p). Der Vorg�nger von Knoten $v_4$ ist $v_2$ und der Vorg�nger von $v_2$ ist $v_1$.(2p) 
Daraus folgt das der Weg vom Startknoten zum Knoten $v_5$ lautet: $v_1\rightarrow v_2\rightarrow v_4\rightarrow  v_5$(1p)  } 
\KlausurKorrekturhinweis{Siehe Musterl�sung.} 
\KlausurErlaeuterung{-}