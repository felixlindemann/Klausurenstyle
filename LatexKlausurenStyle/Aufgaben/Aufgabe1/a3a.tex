% \svnInfo $Id: a3a.tex 2 2011-05-18 10:04:02Z felixlindemann $
				 	 \ifisAufgabenstellung{\newpage	
				 	 }\TeilAufgabe
				 {Beenden Sie das angefangene Verfahren nach Dijkstra. 
				 \ifisAufgabenstellung{Benutzen Sie Tabelle 
				 \ref{tbl.dijkstra.Stud} f�r Ihre L�sung}}
				 {16}			 	
\ifisAufgabenstellung{% \svnInfo $Id: a3tablestud.tex 2 2011-05-18 10:04:02Z felixlindemann $
\begin{table}[H] 
\centering\tiny 
\begin{tabular}{|c|*{9}{|p{0.3cm}|p{0.4cm}|}}
\hline
 & \multicolumn{2}{|c|}{20} & \multicolumn{2}{|c|}{21} & \multicolumn{2}{|c|}{22} & \multicolumn{2}{|c|}{23} & \multicolumn{2}{|c|}{24} & \multicolumn{2}{|c|}{25} & \multicolumn{2}{|c|}{26} & \multicolumn{2}{|c|}{27} & \multicolumn{2}{|c|}{28} \\\hline
 & d & p & d & p & d & p & d & p & d & p & d & p & d & p & d & p & d & p \\\hline\hline
A &  &  &  &  &  &  &  &  &  &  &  &  &  &  &  &  &  & \\[0.5cm]\hline
B &  &  &  &  &  &  &  &  &  &  &  &  &  &  &  &  &  & \\[0.5cm]\hline
C &  &  &  &  &  &  &  &  &  &  &  &  &  &  &  &  &  & \\[0.5cm]\hline
D &  &  &  &  &  &  &  &  &  &  &  &  &  &  &  &  &  & \\[0.5cm]\hline
E &  &  &  &  &  &  &  &  &  &  &  &  &  &  &  &  &  & \\[0.5cm]\hline
F &  &  &  &  &  &  &  &  &  &  &  &  &  &  &  &  &  & \\[0.5cm]\hline
G &  &  &  &  &  &  &  &  &  &  &  &  &  &  &  &  &  & \\[0.5cm]\hline
H &  &  &  &  &  &  &  &  &  &  &  &  &  &  &  &  &  & \\[0.5cm]\hline
I &  &  &  &  &  &  &  &  &  &  &  &  &  &  &  &  &  & \\[0.5cm]\hline
J &  &  &  &  &  &  &  &  &  &  &  &  &  &  &  &  &  & \\[0.5cm]\hline
K &  &  &  &  &  &  &  &  &  &  &  &  &  &  &  &  &  & \\[0.5cm]\hline
L &  &  &  &  &  &  &  &  &  &  &  &  &  &  &  &  &  & \\[0.5cm]\hline
M &  &  &  &  &  &  &  &  &  &  &  &  &  &  &  &  &  & \\[0.5cm]\hline
N &  &  &  &  &  &  &  &  &  &  &  &  &  &  &  &  &  & \\[0.5cm]\hline
O &  &  &  &  &  &  &  &  &  &  &  &  &  &  &  &  &  & \\[0.5cm]\hline
P &  &  &  &  &  &  &  &  &  &  &  &  &  &  &  &  &  & \\[0.5cm]\hline
Q &  &  &  &  &  &  &  &  &  &  &  &  &  &  &  &  &  & \\[0.5cm]\hline
R &  &  &  &  &  &  &  &  &  &  &  &  &  &  &  &  &  & \\[0.5cm]\hline
MUC &  &  &  &  &  &  &  &  &  &  &  &  &  &  &  &  &  & \\[0.5cm]\hline
ECH &  &  &  &  &  &  &  &  &  &  &  &  &  &  &  &  &  & \\[0.5cm]\hline
DAH &  &  &  &  &  &  &  &  &  &  &  &  &  &  &  &  &  & \\[0.5cm]\hline
FFB &  &  &  &  &  &  &  &  &  &  &  &  &  &  &  &  &  & \\[0.5cm]\hline
WES &  &  &  &  &  &  &  &  &  &  &  &  &  &  &  &  &  & \\[0.5cm]\hline
ICK &  &  &  &  &  &  &  &  &  &  &  &  &  &  &  &  &  & \\[0.5cm]\hline
SAU &  &  &  &  &  &  &  &  &  &  &  &  &  &  &  &  &  & \\[0.5cm]\hline
GRA &  &  &  &  &  &  &  &  &  &  &  &  &  &  &  &  &  & \\[0.5cm]\hline
MSC &  &  &  &  &  &  &  &  &  &  &  &  &  &  &  &  &  & \\[0.5cm]\hline\hline
$Q$ & \multicolumn{2}{|c|}{ } & \multicolumn{2}{|c|}{ } & \multicolumn{2}{|c|}{ } & \multicolumn{2}{|c|}{ } & \multicolumn{2}{|c|}{ } & \multicolumn{2}{|c|}{ } & \multicolumn{2}{|c|}{ } & \multicolumn{2}{|c|}{ } & \multicolumn{2}{|c|}{ } \\[1.3cm]\hline 
\end{tabular} 
\caption{Dijkstra Tableau: Iteration 20-28}
\label{tbl.dijkstra.Stud}
\end{table} 
 }
\KlausurErgebnis{\input{Aufgaben/Aufgabe1/a3dijkstraSolution} }
\KlausurMusterLoesung{\input{Aufgaben/Aufgabe1/a3dijkstraSolution}} 
\KlausurKorrekturhinweis{je korrekter Iteration 2p, Keine Folgefehler} 
\KlausurErlaeuterung{-}
%
\ifisAufgabenstellung{\newpage}
\TeilAufgabe
				 {Geben Sie die \textbf{Wegl�ngen} von Eching nach Icking, F�rstenfeldbruck und Grafing an.}
				 {3} 
\KlausurAntwortLinien{4}
\KlausurErgebnis{Eching-Icking = 49\\Eching-F�rstenfeldbruck = 28\\Eching-Grafing = 37 }
\KlausurMusterLoesung{Eching-Icking = 49\\Eching-F�rstenfeldbruck = 28\\Eching-Grafing = 37 } 
\KlausurKorrekturhinweis{je korrekter L�sung 1p, Falls Folgerichtig zur eigenen L�sung, Punkt geben.} 
\KlausurErlaeuterung{-}
%
\TeilAufgabe
				 {Geben Sie die \textbf{Route} sowie die \textbf{Wegl�ngen} von Eching nach We�ling an.}
				 {4} 
\KlausurAntwortLinien{4}
\KlausurErgebnis{Eching-E-D-C-B-We�ling = 31  }
\KlausurMusterLoesung{Eching-E-D-C-B-We�ling = 31   } 
\KlausurKorrekturhinweis{1p f�r Routenl�nge, 3p f�r korrekten Weg, Falls Folgerichtig zur eigenen L�sung, Punkte geben.} 
\KlausurErlaeuterung{-}
%
\TeilAufgabe
				 {Der Knoten $J$ wird �ber den Knoten $Q$ erreicht. 
				 Ein Kollege von Ihnen, der die Strecke Eching-$J$ t�glich f�hrt wundert sich �ber Ihr Ergebnis
				 und wei�t Sie daraufhin, dass er ausgerechnet habe, dass der k�rzeste Weg nicht �ber $Q$ sondern
				 �ber $H$ f�hre. Beziehen Sie zu der Aussage Ihres Kollegen Stellung. Hinweis: Gehen Sie davon aus, 
				 dass die Ihnen gegebene Rechnung korrekt ist.}
				 {4} 
\KlausurAntwortKasten{4cm}
\KlausurErgebnis{Eching-$Q$-$J$ =25+13; Eching-$H$-$J$ = 26+12; Beide Strecken sind gleich lang.}
\KlausurMusterLoesung{Eching-$Q$-$J$ =25+13; Eching-$H$-$J$ = 26+12; Beide Strecken sind gleich lang.} 
\KlausurKorrekturhinweis{2p f�r Rechnung, 2p f�r Erkenntnis \glqq gleich lang\grqq } 
\KlausurErlaeuterung{-}
